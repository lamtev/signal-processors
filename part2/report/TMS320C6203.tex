\documentclass[a4paper,12pt]{extarticle}

\usepackage[utf8x]{inputenc}
\usepackage[T1,T2A]{fontenc}
\usepackage[russian]{babel}
\usepackage{hyperref}
\usepackage{indentfirst}
\usepackage{here}
\usepackage{array}
\usepackage{graphicx}
\usepackage{caption}
\usepackage{subcaption}
\usepackage{chngcntr}
\usepackage{amsmath}
\usepackage{amssymb}
\usepackage{pgfplots}
\usepackage{pgfplotstable}
\usepackage[left=2cm,right=2cm,top=2cm,bottom=2cm,bindingoffset=0cm]{geometry}
\usepackage{multicol}
\usepackage{askmaps}
\usepackage{tikz}
\usepackage{listings}
\usepackage{xcolor}
\usepackage{color}

\addto\captionsrussian{\renewcommand{\contentsname}{Оглавление}}

\newcommand*\circled[1]{\tikz[baseline=(char.base)]{
            \node[shape=circle,draw,inner sep=2pt] (char) {#1};}}

\renewcommand{\not}[1]{\mkern 1.5mu\overline{\mkern-1.5mu#1\mkern-1.5mu}\mkern 1.5mu}
\renewcommand{\le}{\ensuremath{\leqslant}}
\renewcommand{\leq}{\ensuremath{\leqslant}}
\renewcommand{\ge}{\ensuremath{\geqslant}}
\renewcommand{\geq}{\ensuremath{\geqslant}}
\renewcommand{\epsilon}{\ensuremath{\varepsilon}}
\renewcommand{\phi}{\ensuremath{\varphi}}

\lstset{ %
extendedchars=\true,
keepspaces=true,
language=[x86masm]Assembler,						% choose the language of the code
%language=[Motorola68k]Assembler,
morekeywords= {.def, .sect, .cproc, .reg, MVK, SUB.L,%
 .trip, LDW, SHRU, STW, B, .endproc, LDBU, STB, MV},
backgroundcolor=\color{white},   
	basicstyle=\footnotesize\ttfamily,
	commentstyle=\color{green},
	keywordstyle=\color{blue},	
	numberstyle=\footnotesize\color{black},
	stringstyle=\color{purple},	% the size of the fonts that are used for the line-numbers
	numbers=left,
	stepnumber=1,					% the step between two line-numbers. If it is 1 each line will be numbered
	numbersep=5pt,					% how far the line-numbers are from the code
	backgroundcolor=\color{white},	% choose the background color. You must add \usepackage{color}
	showspaces=false				% show spaces adding particular underscores
	showstringspaces=false,			% underline spaces within strings
	showtabs=false,					% show tabs within strings adding particular underscores
	frame=single,           		% adds a frame around the code
	tabsize=2,						% sets default tabsize to 2 spaces
	captionpos=t,					% sets the caption-position to top
	breaklines=true,				% sets automatic line breaking
	breakatwhitespace=false,		% sets if automatic breaks should only happen at whitespace
	escapeinside={\%*}{*)},			% if you want to add a comment within your code
	postbreak=\raisebox{0ex}[0ex][0ex]{\ensuremath{\color{red}\hookrightarrow\space}},
	%texcl=true,
	inputpath=../code-ru,                     % директория с листингами
}

\makeatletter
\newcommand*\@lbracket{[}
\newcommand*\@rbracket{]}
\newcommand*\@colon{:}
\newcommand*\colorIndex{%
\edef\@temp{\the\lst@token}%
\ifx\@temp\@lbracket \color{black}%
\else\ifx\@temp\@rbracket \color{black}%
\else\ifx\@temp\@colon \color{black}%
\else \color{orange}%
\fi\fi\fi
}
\makeatother

\newcommand*\code[1]{\lstinline{#1}}

\counterwithin{figure}{section}
\counterwithin{equation}{section}
\counterwithin{table}{section}
\newcommand{\sign}[1][5cm]{\makebox[#1]{\hrulefill}} % Поля подписи и даты
\graphicspath{{pics/}} % Путь до папки с картинками
\captionsetup{justification=centering,margin=1cm}
\def\arraystretch{1.3}

\begin{document}

\begin{titlepage}
\begin{center}
	САНКТ-ПЕТЕРБУРГСКИЙ ПОЛИТЕХНИЧЕСКИЙ УНИВЕРСИТЕТ\\ ПЕТРА ВЕЛИКОГО\\[0.3cm]
	\par\noindent\rule{10cm}{0.4pt}\\[0.3cm]
	Институт компьютерных наук и технологий \\[0.3cm]
	Кафедра компьютерных систем и программных технологий\\[4cm]
	
	\Large Индивидуальное задание\\ <<Изучение реализации алгоритма Компенсации\\ движения под процессор TMS320C6203>>\\[3mm]
	\Large Дисциплина\\ <<Сигнальные процессоры>>\\[7cm]
\end{center}

\begin{flushleft}
	\hspace*{5mm} Выполнил студент гр. 33501/4  \hspace*{3.85cm}\sign[3cm]\hspace*{2.45mm} А.Ю. Ламтев\\
	\hspace*{11cm} (подпись)\\[3mm]
	\hspace*{5mm} Преподаватель \hspace*{6.8cm}\sign[3cm]\hspace*{2mm} А.В. Лупин\\
	\hspace*{11cm} (подпись)\\[3mm]
	\hspace*{11.1cm} <<\sign[7mm]>> \sign[27mm] \the\year\hspace{1mm} г.
\end{flushleft}

\vfill

\begin{center}
	Санкт-Петербург\\
	\the\year
\end{center}
\end{titlepage}
\addtocounter{page}{1}

\newpage

\section{Алгоритм компенсации движения}

Компенсация движения -- это алгоритмический метод, используемый для прогнозирования кадра в видео на основе данных о предыдущих и/или будущих кадрах путём учёта движения камеры и/или объектов в видеопотоке. Данный алгоритм используется при кодировании видеоданных для сжатия видео, например, в стандарте MPEG-4. Компенсация движения характеризует изображение в терминах преобразования опорного кадра в текущий кадр. Опорный кадр может идти раньше, чем текущий, а может быть и после него. Для видеопотока, в котором последующие кадры могут быть синтезированы на основе ранее переданных, эффективность сжатия может быть улучшена.

\section{Особенности TMS320C6203}

\verb+TMS320C6203+ -- сигнальный процессор с фиксированной точкой, который благодаря архитектуре VelociTI$^{TM}$ обладает очень высокой производительностью и очень длинным словом инструкции (very-long-instruction-word
, VLIW). Благодаря большому параллелизму, на данном процессоре можно эффективно реализовать билинейную интерполяцию, которая используется в алгоритме компенсации движения. Поэтому для решения данной задачи \verb+TMS320C6203+ отлично подходит.

\section{Реализация алгоритма на ассемблере}

В рассматриваемой реализации компенсации движения цветовая модель пикселей -- цветоразностная YUV. Кадр разделяется на макроблоки. Макроблок состоит из 2-х блоков 8 $\times$ 8 пикселей, один из которых для U компоненты, другой для V компоненты, и 4-х блоков 8 $\times$ 8 для Y компоненты. Для простоты кадр разбивается на  8 $\times$ 8 макроблоков.

Реализация алгоритма выглядит следующим образом:

На вход алгоритма передаются:

\begin{itemize}
	\item Указатель на макроблок опорного кадра (\verb+ref_frame+)
	\item Указатель на макроблок текущего кадра (\verb+curr_frame+)
	\item Координаты левого верхнего угла текущего кадра (\verb+position+)
	\item Вектор движения (\verb+motion_vector+)
\end{itemize}

Далее, на основе \verb+motion_vector+ определяется какой способ интерполяции стоит использовать: копирование блока пикселей из опорного кадра в текущий, билинейную интерполяцию в горизонтальном направлении, билинейную интерполяцую в вертикальном направлении или билинейную интерполяцию в обоих направлениях.

Затем \verb+ref_frame+, \verb+curr_frame+, \verb+position+, \verb+motion_vector+ передаются в выбранный алгоритм интерполяции. После интерполяции получаем на выходе \verb+curr_frame+.

В листингах \ref{code:1} -- \ref{code:4} представлен исходный код на ассемблере процессора \verb+TMS320C6203+ для каждого способа интерполяции. 

\lstinputlisting[caption=Копирование блока пикселей, label=code:1]{case_a.asm}
\lstinputlisting[caption=Построчная билинейная интерполяция, label=code:2]{case_b.asm}
\lstinputlisting[caption=Постолбцовая билинейная интерполяция, label=code:3]{case_c.asm}
\lstinputlisting[caption=Билинейная интерполяция в обоих направлениях, label=code:4]{case_d.asm}

\end{document}