\documentclass[a4paper,12pt]{extarticle}

\usepackage[utf8x]{inputenc}
\usepackage[T1,T2A]{fontenc}
\usepackage[russian]{babel}
\usepackage{hyperref}
\usepackage{indentfirst}
\usepackage{here}
\usepackage{array}
\usepackage{graphicx}
\usepackage{caption}
\usepackage{subcaption}
\usepackage{chngcntr}
\usepackage{amsmath}
\usepackage{amssymb}
\usepackage{pgfplots}
\usepackage{pgfplotstable}
\usepackage[left=2cm,right=2cm,top=2cm,bottom=2cm,bindingoffset=0cm]{geometry}
\usepackage{multicol}
\usepackage{askmaps}
\usepackage{tikz}
\usepackage{listings}
\usepackage{xcolor}
\usepackage{color}

\addto\captionsrussian{\renewcommand{\contentsname}{Оглавление}}

\newcommand*\circled[1]{\tikz[baseline=(char.base)]{
            \node[shape=circle,draw,inner sep=2pt] (char) {#1};}}

\renewcommand{\not}[1]{\mkern 1.5mu\overline{\mkern-1.5mu#1\mkern-1.5mu}\mkern 1.5mu}
\renewcommand{\le}{\ensuremath{\leqslant}}
\renewcommand{\leq}{\ensuremath{\leqslant}}
\renewcommand{\ge}{\ensuremath{\geqslant}}
\renewcommand{\geq}{\ensuremath{\geqslant}}
\renewcommand{\epsilon}{\ensuremath{\varepsilon}}
\renewcommand{\phi}{\ensuremath{\varphi}}

\lstset{ %
extendedchars=\true,
keepspaces=true,
language=[x86masm]Assembler,						% choose the language of the code
%language=[Motorola68k]Assembler,
morekeywords= {.def, .sect, .cproc, .reg, MVK, SUB.L,%
 .trip, LDW, SHRU, STW, B, .endproc, LDBU, STB, MV},
backgroundcolor=\color{white},   
	basicstyle=\footnotesize\ttfamily,
	commentstyle=\color{green},
	keywordstyle=\color{blue},	
	numberstyle=\footnotesize\color{black},
	stringstyle=\color{purple},	% the size of the fonts that are used for the line-numbers
	numbers=left,
	stepnumber=1,					% the step between two line-numbers. If it is 1 each line will be numbered
	numbersep=5pt,					% how far the line-numbers are from the code
	backgroundcolor=\color{white},	% choose the background color. You must add \usepackage{color}
	showspaces=false				% show spaces adding particular underscores
	showstringspaces=false,			% underline spaces within strings
	showtabs=false,					% show tabs within strings adding particular underscores
	frame=single,           		% adds a frame around the code
	tabsize=2,						% sets default tabsize to 2 spaces
	captionpos=t,					% sets the caption-position to top
	breaklines=true,				% sets automatic line breaking
	breakatwhitespace=false,		% sets if automatic breaks should only happen at whitespace
	escapeinside={\%*}{*)},			% if you want to add a comment within your code
	postbreak=\raisebox{0ex}[0ex][0ex]{\ensuremath{\color{red}\hookrightarrow\space}},
	%texcl=true,
	inputpath=../code-ru,                     % директория с листингами
}

\makeatletter
\newcommand*\@lbracket{[}
\newcommand*\@rbracket{]}
\newcommand*\@colon{:}
\newcommand*\colorIndex{%
\edef\@temp{\the\lst@token}%
\ifx\@temp\@lbracket \color{black}%
\else\ifx\@temp\@rbracket \color{black}%
\else\ifx\@temp\@colon \color{black}%
\else \color{orange}%
\fi\fi\fi
}
\makeatother

\newcommand*\code[1]{\lstinline{#1}}

\counterwithin{figure}{section}
\counterwithin{equation}{section}
\counterwithin{table}{section}
\newcommand{\sign}[1][5cm]{\makebox[#1]{\hrulefill}} % Поля подписи и даты
\graphicspath{{pics/}} % Путь до папки с картинками
\captionsetup{justification=centering,margin=1cm}
\def\arraystretch{1.3}